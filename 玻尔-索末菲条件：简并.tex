\documentclass{ctexart}

\title{玻尔-索末菲条件:简并}
\author{}
\date{}

\usepackage{amsmath}
\usepackage{amsfonts}
\usepackage{amssymb}

\catcode`\。=13
\newcommand{。}{\ifmmode\text{.}\else.\fi}
\let\commasaved=,
\catcode`\,=13
\newcommand{,}{\ifmmode\text{\commasaved}\else\commasaved\fi}

\begin{document}

\maketitle

我们称“同一能量对应不同的态”为\textbf{简并}。不同偏心率的椭圆轨道的能量可以相同,也就是简并。玻尔的原子模型只考虑了圆形轨道,因而没有考虑简并问题。

椭圆轨道径矢不像圆周运动中那样是常数。在玻尔-索末菲条件
\[
\oint p\mathrm{d}r=nh
\]
中,我们选择径矢$r$和径向动量$p_{r}$这组正则共轭变量,相应的作用量(径向作用量)为$\oint p_{r}\mathrm{d}r$。

首先计算$p_{r}$。有心力场中的系统总能量表示为
\[
\begin{aligned}
E & =\frac{1}{2}mv^{2}-\frac{Ze_{s}^{2}}{r}\\
& =\frac{p_{r}^{2}}{2m}+\underbrace{\frac{L^{2}}{2mr^{2}}}_{\text{离心势能}}-\frac{Ze^{2}_{s}}{r},
\end{aligned}
\]
其中$Z$是原子序数。有心力场中角动量守恒,因此$E$和$L$均为常数,由此可解出
\[
\begin{aligned}
p_{r} & =\sqrt{2m\left(E+\frac{Ze_{s}^{2}}{r}-\frac{L^{2}}{2mr^{2}}\right)}\\
& =\sqrt{2m\left(E+\frac{Ze^{2}_{s}}{r}\right)-\frac{L^{2}}{r^{2}}}。
\end{aligned}
\]
进一步计算作用量:
\[
\begin{aligned}
\oint p_{r}\mathrm{d}r & =\oint\sqrt{2m\left(E+\frac{Ze_{s}^{2}}{r}\right)-\frac{L^{2}}{r^{2}}}\mathrm{d}r\\
& =2\int_{r_{\min}}^{r_{\max}}\sqrt{2m\left(E+\frac{Ze_{s}^{2}}{r}\right)-\frac{L^{2}}{r^{2}}}\mathrm{d}r,
\end{aligned}
\]
这里的$r_{\max}$和$r_{\min}$由条件
\[
p_{r}\left(=m\dot{r}\right)=0
\]
确定。由
\[
\sqrt{2m\left(E+\frac{Ze^{2}_{s}}{r}\right)-\frac{L^{2}}{r^{2}}}=0
\]
并注意到$E<0$(束缚态),可解得
\[
r_{\max}=-\frac{1}{2E}\left(Ze_{s}^{2}+\sqrt{2E\frac{L^{2}}{m}+Z^{2}e_{s}^{4}}\right),
\]
\[
r_{\min}=-\frac{1}{2E}\left(Ze_{s}^{2}-\sqrt{2E\frac{L^{2}}{m}+Z^{2}e_{s}^{4}}\right)。
\]
代入积分可得
\[
\oint p_{r}\mathrm{d}r=-2\pi L+2\pi Ze^{2}\sqrt{\frac{m}{-2E}},
\]
解出$E$为
\[
E=-\frac{m}{2}\frac{Z^{2}e_{s}^{4}}{\left(\frac{J_{r}}{2\pi}+L\right)^{2}}。
\]
由量子化条件
\[
\oint p_{r}\mathrm{d}r=sh
\]
(其中$s$是径向量子数,是整数)和玻尔的量子化条件
\[
L=l\hbar,
\]
有
\[
E=-\frac{m}{2}\frac{Z^{2}e_{s}^{4}}{\hbar^{2}}\frac{1}{(l+s)^{2}},
\]
一般我们取$n=l+s$,并称$n$为主量子数。

由这个结果可以看到,$n$确定则能量$E$也确定,但在能量$E$确定的前提下仍有许多不同的态,分别对应$l$和$s$的各种取值。这就是说,氢原子的轨道是简并的。在半经典图像中,这些能量相同的轨道对应偏心率不同的椭圆轨道。
\end{document}
